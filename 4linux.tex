\documentclass[12pt]{article}
\usepackage[portuguese]{babel}
\usepackage[utf8]{inputenc}
\usepackage[T1]{fontenc}
\usepackage{hyphenat}
\hyphenation{mate-mática recu-perar}
\usepackage{amsmath}
\usepackage{graphicx}
\usepackage[colorinlistoftodos]{todonotes}
\usepackage{hyperref}
\usepackage{listings}
\usepackage[brazilian,hyperpageref]{backref}
\usepackage[alf]{abntex2cite}
\usepackage{minted}
\begin{document}
\begin{titlepage}

    \newcommand{\HRule}{\rule{\linewidth}{0.5mm}} % Defines a new command for the horizontal lines, change thickness here

        \center % Center everything on the page

        %----------------------------------------------------------------------------------------
        %	HEADING SECTIONS
        %----------------------------------------------------------------------------------------

        \textsc{\LARGE 4linux - Open Software Specialists}\\[1.5cm] % Name of your university/college
        \textsc{\Large Processo de Seleção - Dezembro/2016}\\[0.5cm] % Major heading such as course name
        \textsc{\large Desafio Técnico}\\[0.5cm] % Minor heading such as course title

        %----------------------------------------------------------------------------------------
        %	TITLE SECTION
        %----------------------------------------------------------------------------------------

        \HRule \\[0.4cm]
        { \huge \bfseries Automatização de VirtualHosts e Instâncias Web}\\[0.4cm] % Title of your document
        \HRule \\[1.5cm]

        %----------------------------------------------------------------------------------------
        %	AUTHOR SECTION
        %----------------------------------------------------------------------------------------

        \begin{minipage}{0.4\textwidth}
            \begin{flushleft} \large
                \emph{Autor:}\\
                Tiago R. \textsc{Assunção} % Your name
            \end{flushleft}

        \end{minipage}
        ~
        \begin{minipage}{0.4\textwidth}
            \begin{flushright} \large
                \emph{Supervisor:} \\
                Rodrigo \textsc{Tornis} % Supervisor's Name
            \end{flushright}
        \end{minipage}\\[2cm]
        {\large \today}\\[2cm] % Date, change the \today to a set date if you want to be precise
        \vfill % Fill the rest of the page with whitespace
\end{titlepage}
\section{Introdução}
Para o desafio técnico, foi solicitado executar a automatização de virtualhosts e instâncias Web utilizando puppet. Para a implementação da solução,
utilizou-se um módulo desenvolvido pela Puppet Labs, chamado
\href{https://github.com/puppetlabs/puppetlabs-apache}{PuppetLabs Apache}. 

Ao longo deste relatório será demonstrado os passos necessários para implementar a solução final.

\section{Configuração Manual}
\label{sec:configura_o_manual}
Antes do processo de automatização, foi decidido testar a solução manual, a fim de detectar todos os possíveis problemas durante a configuração.
A verificação deu-se em um host diferente do cedido pela 4linux. O servidor utilizado é apontado pelo domínio \href{networkabout.com}{networkabout}.

Dessa maneira, foram criadas as respectivas pastas dentro do diretório /var/www/:

\begin{enumerate}
    \item teste.networkabout.com/public\_html/index.html 
    \item exemplo.networkabout.com/public\_html/index.html 
\end{enumerate}

Foram dadas permissões para que o usuário atual passasse a ter direitos de owner dos diretórios e garantir que estes tenham acesso de leitura
permitido.

\begin{minted}[mathescape,
    linenos,
               numbersep=5pt,
               gobble=2,
               frame=lines,
               framesep=2mm]{bash}

    sudo chown -R $USER:$USER /var/www/examplo.networkabout.com/public_html
    sudo chown -R $USER:$USER /var/www/teste.networkabout.com/public_html
    sudo chmod -R 755 /var/www

\end{minted}

Logo em seguida, os arquivos de configuração foram criados. Eles são bem semelhantes, tendo mudanças apenas nos nomes, entre teste e exemplo.
O arquivo /etc/apache2/sites-available/exampleo.networkabout.com.conf está descrito em seguida:

\begin{minted}[mathescape,
    linenos,
               numbersep=5pt,
               gobble=2,
               frame=lines,
               framesep=2mm]{apache}
    Listen 81
    NameVirtualHost *:81	
    <VirtualHost *:81>
            ServerAdmin tiago@sof2u.com
            ServerName exemplo.networkabout.com
            ServerAlias www.exemplo.networkabout.com
            DocumentRoot /var/www/exemplo.networkabout.com/public_html
            ErrorLog ${APACHE_LOG_DIR}/error.log
            CustomLog ${APACHE_LOG_DIR}/access.log combined
    </VirtualHost>

    # vim: syntax=apache ts=4 sw=4 sts=4 sr noet
\end{minted}

A outra configuração é semelhante, alterando apenas os nomes e a porta para 80. Dessa forma, os arquivos de configuração
foram habilitados com os seguintes comandos:

\begin{minted}[mathescape,
    linenos,
               numbersep=5pt,
               gobble=2,
               frame=lines,
               framesep=2mm]{bash}
    sudo a2ensite examplo.networkabout.com.conf
    sudo a2ensite teste.networkabout.com.conf
\end{minted}

Para finalizar, o servidor do apache foi reiniciado e os vhosts se comportaram como o esperado.

Para executar os testes manuais, foram criados os domínios do tipo A. Sendo assim, não houve necessidade de configurar nenhum domínio
em /etc/hosts.

\section{Processo de Automatização}
\label{sec:processo_de_automatiza_o}
Para automatizar o processo com Puppet, foi estabelecida a escolha de utilizar o módulo desenvolvido para
manutenção do Apache, sendo que este tem a opção de lidar com vhosts. Dessa forma, primeiramente, instalou-se o módulo:

\begin{minted}[mathescape,
    linenos,
               numbersep=5pt,
               gobble=2,
               frame=lines,
               framesep=2mm]{bash}
    puppet module install puppetlabs-apache
\end{minted}

O primeiro passo da instalação foi automatizar a criação dos arquivos HTML nos diretórios corretos. Para tanto,
implementou-se uma classe que cria apenas o arquivo e os diretórios de um dos apps.

\begin{minted}[mathescape,
    linenos,
               numbersep=5pt,
               gobble=2,
               frame=lines,
               framesep=2mm]{puppet}
    $app1 = 'app1.linux.com.br'

    class create_index{
        # create a directory      
      file {["/var/www/${app1}", "/var/www/${app1}/public_html"]:

        ensure => 'directory',
        recurse => true,
      }

      file { "/var/www/${app1}/public_html/index.html":
        ensure  => 'present',
        replace => 'no', # this is the important property
        content => "<h1>INSTANCIA1</h1>",
        mode    => '0644',
      }
  }

    include create_index
\end{minted}

Para que fosse possível enviar parâmetros ao trecho de código que seria executado para a criação dos diretórios e arquivos,
foi definida uma função que recebe dois parâmetros: domínio e index.

\begin{minted}[mathescape,
    linenos,
               numbersep=5pt,
               gobble=2,
               frame=lines,
               framesep=2mm]{puppet}
    $app1 = 'app1.linux.com.br'

    define create_index($domain, $index){
        # create a directory      
      file {["/var/www/${domain}", "/var/www/${domain}/public_html"]:
        replace => 'yes', # this is the important property
        ensure => 'directory',
        recurse => true,
      }

      file { "/var/www/${domain}/public_html/index.html":
        ensure  => 'present',
        replace => 'yes', # this is the important property
        content => "<h1>INSTANCIA1</h1>",
        mode    => '0644',
      }
  }

    create_index{
        "1" : domain => $app1, index => 1;
    }
\end{minted}

Foi observado que não era necessário gerar variáveis ou arrays para a criação dos apps, mas sim, apenas um índice para numerá-los.
Assim, o código foi escrito da seguinte forma:

\begin{minted}[mathescape,
    linenos,
               numbersep=5pt,
               gobble=2,
               frame=lines,
               framesep=2mm]{puppet}
    define create_index($index){
        # create a directory      
      file {[
          "/var/www/app${index}.4linux.com.br",
        "/var/www/app${index}.4linux.com.br/public_html"]:
    ensure => 'directory',
          recurse => true,
      }

      file { "/var/www/app${index}.4linux.com.br/public_html/index.html":
        ensure  => 'present',
        replace => 'yes',
        content => "<h1>INSTANCIA${index}</h1>",
        mode    => '0644',
      }
  }

    create_index{
        "app1" : index => 1;
      "app2" : index => 2;
      "app3" : index => 3;
  }
\end{minted}

Além de deixar o código mais limpo, este processo permite que a mesma função seja chamada outras vezes para os
aplicativos consecutivos, como podemos ver nas linhas 20 e 21.

O passo consecutivo foi a criação dos vhosts utilizando o módulo do apache. Um exemplo de um vhost criado segue:


\begin{minted}[mathescape,
    linenos,
               numbersep=5pt,
               gobble=2,
               frame=lines,
               framesep=2mm]{puppet}
    apache::vhost { 'app2.4linux.com.br':
      port    => '81',
      docroot => '/var/www/app1.4linux.com.br/public_html',
      docroot_owner => 'www-data',
      docroot_group => 'www-data',
      options       => ['Indexes','FollowSymLinks','MultiViews'],
      override      => ['none'],
      proxy_pass => [
          {
              'path' => '/app2',
          'url'  => 'http://localhost:81/app2'
      },
  ],
    }
\end{minted}

Para evitar a duplicação de código, foi definida a função set\_vhost, que recebe a porta que o vhost irá operar e o
seu respectivo index.

\begin{minted}[mathescape,
    linenos,
               numbersep=5pt,
               gobble=2,
               frame=lines,
               framesep=2mm]{puppet}
    define set_vhosts($port, $index){
        apache::vhost { 'app1.4linux.com.br':
        port    => $port,
        docroot => "/var/www/app${index}.4linux.com.br/public_html",
        docroot_owner => 'www-data',
        docroot_group => 'www-data',
        options       => ['Indexes','FollowSymLinks','MultiViews'],
        override      => ['none'],
      }
  }

    set_vhosts{
        "app1" : port => 80, index => 1;
      "app2" : port => 81, index => 2;
      "app3" : port => 82, index => 3;
  }
\end{minted}

Foi executado um refinamento nas opções do apache::vhost e verificou-se que a chamada de algumas opções
eram redundantes, como override, docroot\_owner e docroot\_group. Dessa maneira, implementou-se  a última
refatoração, eliminando estas configurações desnecessáiras.


\begin{minted}[mathescape,
    linenos,
               numbersep=5pt,
               gobble=2,
               frame=lines,
               framesep=2mm]{puppet}
    define set_vhosts($port, $index){
        apache::vhost { "app${index}.4linux.com.br":
        servername    =>  "app${index}.4linux.com.br",
        port          => $port,
        docroot       => "/var/www/app${index}.4linux.com.br/public_html",
        options       => ['Includes'],
        require => Exec['reload'],
      }
  }
\end{minted}

Para finalizar, foi adicionado um exec ao final de cada vhost, para executar o reload do apache:

\begin{minted}[mathescape,
    linenos,
               numbersep=5pt,
               gobble=2,
               frame=lines,
               framesep=2mm]{puppet}
    exec { "reload":
      command => "/etc/init.d/apache2 reload",
      user => "root",
    }
\end{minted}


O código completo, bem como cada passo do desenvolvimento pode ser visto no repositório do 
\href{https://github.com/TiagoAssuncao/set-vhost-puppet/}{GitHub}. 

Todo o procedimento foi executado e testado na VPS de IP 162.243.186.92. Para o processo automatizado,
os domínios foram editados no arquivo /etc/hosts da seguinte maneira:

\begin{minted}[mathescape,
    linenos,
               numbersep=5pt,
               gobble=2,
               frame=lines,
               framesep=2mm]{c}
    127.0.0.1       localhost
    127.0.1.1       teslla.teslla   teslla
    127.0.1.1       aboutmeet.com   teslla
    162.243.186.92  app1.4linux.com.br
    162.243.186.92  app2.4linux.com.br
    162.243.186.92  app3.4linux.com.br
\end{minted}

Para os testes, foram inseridas as urls seguintes, com os respectivos resultados:

\begin{itemize}
    \item http://app1.4linux.com.br/
    \item http://app2.4linux.com.br:81/
    \item http://app3.4linux.com.br:82/
\end{itemize}

\begin{itemize}
    \item INSTANCIA1a
    \item INSTANCIA2a
    \item INSTANCIA3a
\end{itemize}

Para iniciar a configuração de infraestrutura utilizando o código desenvolvido, basta apenas
executar o puppet com previlégios de super usuário:

\begin{minted}[mathescape,
    linenos,
               numbersep=5pt,
               gobble=2,
               frame=lines,
               framesep=2mm]{bash}
    sudo puppet apply puppet-apache.pp
\end{minted}

\section{Conclusão}
\label{sec:conclus_o}

Utilizando a automatização de Deploy, é possível configurar a infraestrutura como um código, estruturado e testado
para seguir um padrão estabelecido. Ao executar a configuração de ambiente uma vez, esta pode ser repetida quantas
vezes forem necessárias, possibilitando a alta escalabilidade. Além de, existir uma vasta documentação para o Puppet, 
facilitando o trabalho do desenvolvedor.

Caso a tecnologia Puppet não seja de agrado da equipe, existem várias outras ferramentas alternativas que fazem bem o
trabalho, como: \href{https://www.chef.io/chef/}{Chef}, \href{http://cfengine.com/}{CFEngine}, \href{http://saltstack.com/}{Salt},
\href{http://www.ansibleworks.com/}{Ansible}. De toda maneira, são muitas as opções que possibilitam a automatização
de infraestrutura com êxito.
\end{document}
